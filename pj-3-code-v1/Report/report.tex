\documentclass{article} % This command is used to set the type of document you are working on such as an article, book, or presenation
\usepackage{geometry} % This package allows the editing of the page layout
\usepackage{amsmath}  % This package allows the use of a large range of mathematical formula, commands, and symbols
\usepackage{graphicx}  % This package allows the importing of images

\newcommand{\question}[2][]{\begin{flushleft}\textbf{Question #1}: \textit{#2}\end{flushleft}}
\newcommand{\sol}{\textbf{Solution}:} %Use if you want a boldface solution line
\newcommand{\maketitletwo}[2][]{\begin{center}
        \Large{\textbf{Project 4 Report}
        
            Theory of Computer Game} % Name of course here
        \vspace{5pt}
        
        \normalsize{
            Name: Kai-Jie Lin 
            
            Student ID: 110652019
            
            \today}
        \vspace{15pt}
        \end{center}}
\begin{document}
    \maketitletwo[5]  % Optional argument is assignment number
    %Keep a blank space between maketitletwo and \question[1]
    
    \section{Method and Improvement} 
    The method I use for this game is monte carlo tree search with root parallelization. Total number of simulation is set to 
    40000 and every thinking time is limited to 10 seconds in order not to exceed the total thinking time 300 seconds in a game.
    \subsection*{Time Management}
    Use the method in paper Time Management for Monte Carlo Tree Search by Hendrik Baier and Mark H. M. Winands, Member, IEEE.
    I have used some method mentioned in the paper which is EXP-MOVE with STOP. EXP-MOVE make the agent to spend greater time to 
    search at first then little time at the end. And STOP set the criterion to early stop grow the tree. 
    The termination criterion of STOP is: $ \frac{n \cdot timeleft_n}{timespent_{n}} \cdot p_{earlystop} \le visits_{best_{n}} - vists_{secondbest_{n}}$
    The constant $p_{earlystop}$ is set to 0.9. The method I have used in tournament in the bigining, but I found it is bad in the tournament.
    Since there are a lot of time to search in the begin, but it is not simulation in the best way. Thus there is no time for the end, then the agent would play at random.
    \subsection*{Root Parrelization}
    I used 4 threads to run root parallelization, since I had to use tcglinux machine. I found that if there are 10~20 threads would be better, but my computer is Mac QQ.
    \subsection*{RAVE}
    I also used RAVE in the tournament, but I can't determine it is work or not.
\end{document}